\chapter{Introdução}
A sociedade está lidando com uma quantidade de dados cada vez maior. 
% Hoje, por menor que seja o dado, ele se torna importante pelas informações que podem ser extraídas a partir dele. Se pararmos para pensar, estamos envolvidos por uma quantidade de dados enorme. Como a necessidade de extrair informação é comum em um mundo globalizado e informatizado,  os cientistas e engenheiros se veem obrigados a desenvolvem novas maneiras de medir eventos. Sensores, câmeras de trânsito, dados da web, genes, dados geográficos, dados de compras, dados de pesquisas, e muitas outras informações se tornaram o diferencial nesse mundo competitivo e dinâmico, e precisam de tratamento e atenção~\cite{initBigData}. 
Conforme Borkar et al. ~\cite{WNextBigData} descreve, as empresas passaram a monitorar compras de clientes, pesquisas de produtos, sites de relacionamento e diversas outras fontes para aumentar a eficácia do seu marketing e dos serviços ofertados aos clientes; governos e empresas estão rastreando conteúdos de \textit{blogs} e \textit{tweets} para realizar análises de sentimentos; e organizações públicas de saúde estão monitorando artigos de notícias, \textit{tweets}, e tendências de pesquisas na web para acompanhar o progresso de epidemias~\cite{WNextBigData}. Esse grande volume de dados disponíveis e gerenciados leva a muitos desafios tanto para a academia quanto para a sociedade em geral.

Importante destacar que a definição de ``grande volume de dados'' evoluiu significativamente. Há pouco tempo, por exemplo, o armazenamento na ordem de \emph{terabytes} era algo restrito a poucos domínios de aplicação (como os domínios de telefonia e financeiro). Por outro lado, atualmente a maior parte dos dispositivos de armazenamento alcançam capacidades superiores a um \emph{terabyte} e  empresas dos mais variados segmentos já gerenciam volumes de dados da ordem de \emph{petabytes}.
Isso também inclui órgãos da administração pública, que atualmente mantêm a documentação do funcionalismo público em pastas físicas e que precisam ser armazenados com qualidade e cuidado, pois fazem parte dos chamados arquivos permanentes. Esses arquivos ocupam cada vez mais espaço e, devido a sua característica, devem ser preservados por um longo período de tempo. Por outro lado, inúmeras vezes os órgãos precisam consultar esses arquivos, em processos que consomem tempo, são difíceis de se realizar e contribuem para a deteriorização dos documentos~\cite{arqConarq}.

A dificuldade em manter fisicamente esses documentos fez com que o governo federal incentivasse a administração pública a iniciar um processo de digitalização dos documentos~\cite{portariaMP} para que uma cópia digital desses arquivos fosse mantida pelos órgãos\footnote{Mais informa\c c\~{o}es sobre essa iniciativa podem ser encontradas na Portaria Normativa MP 3, de 18 de Novembro de 2011}. A digitalização dos arquivos não só possibilita a preservação dos documentos, pois restringe o manuseio dos originais, quanto também facilita o acesso, já que torna mais efetivo os acessos locais, remotos e/ou simultâneos. O processo de digitalização é complexo, demorado e, além de um controle de \textit{work flow} bem definido, necessita de grandes investimentos de \textit{software} e \textit{hardware} para que o resultado tenha uma boa qualidade~\cite{arqConarq}. Em linhas gerais, o conceito de documentos descentralizados em pastas funcionais físicas será substituído por repositórios de dados e informações de origem primária, auditáveis e não replicados. Isso caracteriza o Projeto de Assentamento Funcional Digital – AFD, que objetiva a criação de um \emph{dossiê}, em mídia digital, que será tratado como Fonte Primária de Informação de dados cadastrais do Servidor Público Civil Federal e que substituirá a tradicional Pasta Funcional ou Assentamento Funcional. No site do SIGEPE (Sistema de Gestão de Pessoas) ~\cite{siteSIGEPE} são destacados alguns pontos de melhoria com a criação do AFD, destacando que ``\emph{A criação do Assentamento Funcional Digital (AFD) possibilitará a diminuição drástica do volume de papeis armazenados e tramitados. O AFD constituirá de um banco referencial, de dados e imagens das pastas funcionais, com indexadores para localização dos documentos de maneira online}''~\cite{apresentAFD}.


Para gerenciar grandes volumes de dados (caracterizando ambientes de \emph{big data}), Podemos dividir as tecnologias em duas classes: as  tecnologias envolvidas com análise dos dados, como \emph{Hadoop} e o modelo de computa\c c\~{a}o \emph{MapReduce}; e as tecnologias de armazenamento eficiente para grandes volumes de dados~\cite{ibmvcsabeoqebigdata}, cujos avanços recentes levaram ao surgimento dos bancos de dados \emph{NoSQL} (\emph{Not only SQL}), com inovações relacionadas não apenas ao armazenamento mas também a distribuição de dados. Em linhas gerais, os Sistemas Gerenciadores de Bancos de Dados (SGBDs) relacionais não garantiam o tempo de resposta e escalabilidade esperados para ambientes de \emph{big data}, fazendo com que os modelos \emph{não relacionais} implementados por bancos de dados \emph{NoSQL} passassem a ter uma aceitação crescente.
 
\section{Objetivos e Justificativa}

O principal objetivo desse trabalho é comparar os modelos relacional e não relacional de armazenamento para o contexto do Assentamento Digital Funcional (AFD). Mais especificamente, como existem diferentes alternativas de armazenamento em bancos de dados não relacionais, esse trabalho utiliza o modelo orientado a documentos como representante da \emph{classe} de bancos de dados não relacionais. 

Para atingir o objetivo principal, tínhamos os seguintes objetivos específicos:

\begin{itemize}
\item compreender, abstrair e modelar os conceitos e operações do AFD.
\item modelar os conceitos do AFD utilizando a estratégia relacional.
\item modelar os conceitos do AFD utilizando a estratégia orientada a documentos.
\item implementar os modelos em SGBDs relacionais e orientados a documentos.
\item implementar uma arquitetura SOA (\emph{Service Oriented Architecture})para realizar as operações do AFD, utilizando para persitência tando um SGBD relacional quanto um SGBD não relacional.
\item projetar, implementar e realizar testes de desempenho considerando operações e volumes de dados que permitam tirar conclusões sobre quais dos modelos são mais propícios para o armazenamento dos dados do AFD. 
\end{itemize}

Para cumprir esses objetivos,  a arquitetura proposta para sustentar os testes consiste em serviços~\cite{erl:2007} com capacidades simples de inserção, consulta, exclusão e atualização de dados; utilizando uma camada de persistência implementada em diferentes SGBDs. Os testes de desempenho da solução usam um banco de dados relacional (PostgresSQL) ou um banco de dados orientado a documentos (MongoDB). 

Essa investigação se justifica porque, para que a base de dados do AFD possa cumprir com o seu propósito, ela precisa garantir bom tempo de resposta e escalabilidade. 


\section{Organização do Documento}

Para um melhor aproveitamento do estudo aqui apresentado, o trabalho foi organizado da seguinte maneira:

\begin{enumerate}

\item No capítulo dois serão apresentados os principais conceitos relavantes ao trabalho: \textit{Big Data},\textit{NoSQL}, \textit{Teste de Software} e \textit{Web services}

\item O capítulo três será responsável por mostrar como o protótipo dos testes foi construído e o que foi feito para permitir alcançar os objetivos propostos.

\item No capítulo quatro será apresentado e discutido sobre os resultados obtidos nos testes.

\item O capítulo cinco é composto por algumas considerações finais a respeito do trabalho e são enumerados alguns possíveis projetos futuros.

\end{enumerate}

