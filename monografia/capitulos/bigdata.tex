\chapter{Big Data}


A quantidade de informação que está disponível para a humanidade é enorme e a medida que o conhecimento humano se expande, maior é a quantidade dessa informação que precisa ser armazenada e analizada. Além da quantidade, o fluxo e variedade dessas informações constantemente desafiam a indústria e a academia a medida em que a quantidade de Big Data aumenta exponencialmente. Nesse capítulo veremos uma definição detalhada de o que é Big Data, as tecnologias que apoiam esse domínio.

\section{O que é Big Data?}

Em um estudo divulgado em 2011 o tamanho do universo digital quebrou a barreira dos zettabytes e esse número está crescendo rapidamente~\cite{emcuniversedigital}. Cientistas de diversas áreas estão vendo o grande potencial de conhecimento que se pode adiquirir pela análise a armazenamento de informação digital. Conforme já dito anteriormente o conceito de 'grande (big)'  foi mudando no decorrer da nossa história. Na década de 70, grande significada megabytes; ao longo do tempo cresceu para gigabytes e em seguida, a terabytes. Atualmente já podemos dizer que grande é petabyte e até mesmo exabytes~\cite{WNextBigData}. Contudo  o  conceito de Big Data não se dá somente por tamanho ou domínio, mas sim por um conjunto de características que o difere de uma base de dados comum.

Segundo Gartner big data é definido, em geral, como uma massa de dados de grande volume, velocidade e variedade de informações que exigem formas inovadoras de processamento para maior visibilidade e tomada de decisão~\cite{conceitoGartner}. A maioria dos estudiosos compartilham dessa mesma definição e dizem que Big Data é caracterizado por no mínimo três V's. Volume, variedade e  velocidade.~\cite{ibmbigdatavvv,fromdbtobigdata}

Volume é a característica mais fácil de se perceber. Geramos enormes quantidades de dados todos os dias, e essa quantidade só tende a aumentar. Redes sociais, dispositivos móveis que guardam nossas informações, sites que armazenam nossas preferencias, dispositivos de busca que indexam as páginas da web e a popularização da computação em nuvem nos colocam em uma época de grande volume de dados, uma época em que tudo é informação, tudo é valioso, tudo pode ser extraído. Cada dia fica mais comum grandes empresas terem de ligar com dados na ordem de petabytes. Variedade é outra característica que é de fácil percepção, pois os dados são de diversas naturezas como email, dados gerados por mídias sociais (blogs, Twitter, Youtube, Facebook, Wikis), documentos eletrônicos, apresentações, fotos, mensagens instantaneas, dados médicos, videos, etc. A característica de velocidade é explicada quando precisamos processar os dados praticamente em tempo real como em controle de tráfego, detecções de fraudes e propagandas dinâmicas na web. Os dados são cada vez mais usados para tomadas de decisão em tempo real~\cite{promiseperil}.

\begin{table}
	\caption{Tabela de bytes}
	\begin{center}
	\begin{tabular}{ccc}
		\hline
			\textbf{Nome} & \textbf{Tamanho} & \textbf{Abreviação} \\
		\hline
			\texttt{Kilobyte}	& $10^3$ & KB \\
			\texttt{Megabyte}	& $10^6$ & MB \\
			\texttt{Gigabyte}	& $10^9$ & GB \\
			\texttt{Terabyte}	& $10^{12}$ & TB \\
			\texttt{Petabyte}	& $10^{15}$ & PB \\
			\texttt{Exabyte}	& $10^{18}$ & EB \\
			\texttt{Zettabyte}	& $10^{21}$ & ZB \\
			\texttt{Yottabyte}	& $10^{24}$ & YB \\
		\hline
	\end {tabular}
	\end{center}
	\label{tab:bytes}
\end{table}

Dada a problemática do armazenamento, ao se deparar com os limites de técnicas e ferramentas disponíveis o mercado tratou de criar suas próprias soluções de gerenciamento de dados, em sua maioria não relacional. Usando a tecnologia apropriada, profissionais capacitados podem transformar grandes massas de dados em informações muito valiosas. Muitos sistemas comercias relacionais se dizem capazes de lidar com vários petabytes de base de dados (Greenplum,Netezza,Teradata, ou Vertica). Apesar dessa quantidade de dados atender a grande maioria das empresas, existem empresas de grande porte como o Google e o Facebook que não são atendidas e precisaram criar suas próprias soluções, além disso, sistemas open source como Postgres  não tem o mesmo nível de escalabilidade que os comerciais ~\cite{fromdbtobigdata}.
%Dado isso, foram surgindo alternativas ao Modelo Relacional das quais a que mais se destacou foi o paradigma NoSql.(tcc-gleison)

\section{Tecnologias de Apoio}
\subsection{Bases Relacionais}

Quando pensamos em armazenamento de dados em SGBDs logo associamos essa idéia ao método tradicional que inclui bancos de dados como MySQL, PostgreSQL, modelagem relacional e esquemas de dados bem definidos. O modelo de dados relacional foi introduzido por Ted Codd, da IBM Research, em 1970, em um artigo que conseguiu atrair grande atenção devido a simplicidade e base matemática. Os SGBDs relacionais mais populares atualmente são o DB2 e Informix Dynamic Server (IBM), o Oracle e Rdb (Oracle), o Sybase SGBD (Sybase) e o SQLServer e Access (Microsoft). Ainda temos os de código aberto como o MySQL e PostgreSQL.

O modelo relacional representa o banco de dados como uma coleção de relações. Uma relação é como se fosse uma tablea de valores ou um arquivo de registros. Cada tabela é formada por uma ou mais colunas de dados. Por sua vez, cada linha na tabela contém uma instância única de dado para as categorias de colunas definidas. No modelo relacional é possível criar conexões entre as tabelas e os  campos e os formatos dos valores são bem definidos, ou seja, possui um schema de dados~\cite{SBElmasri,nosqlliveup}.

Os SGBDs (Sistemas Gerenciadores de Banco de Dados) relacionais proveem diversas garantias aos seus usuários como: validação, verificação e garantias de integridade dos dados, controle de concorrência, recuperação de falhas,  entre outros. As propriedades que esses bancos têm de prover são a atomicidade, consistência, isolamento e a durabilidade (ACID).Todas essas características mantém os SGBDs como principal solução na maioria dos ambientes computacionais, mas não impediram o surgimento de problemas, em alguns casos, causados pela rígida estrutuda definida pelo layout das tabelas, nomes e tipos das colunas.

As abordagens mais usadas para manipular grandes bases nesse tipo de estrutura são os data warehouses e data marts. Um data warehouse é um banco de dados relacional usado para armazenar, analizar e gerar relatórios sobre os dados. O data mart é a camada usada para acessar o data warehouse. As duas abordagem usadas para se armazenar dados em um data warehouse são a normalização e a modelagem dimensional~\cite{bigdataarchitectureandapproach}.

\subsubsection{Limitações}

Com a evolução das aplicações e com requisitos cada vez mais exigentes, foram surgindo casos em que os banco de dados relacionais não escalavam. Operações de joins estão presentes nos menos dos bancos de dados relacionais, e esse tipo de operação é lenta.Para que SGBDs relacionais consigam garantir consistência para os dados eles usam o conceito de transações, o que requer um bloqueio nos dados durante um certo período de tempo.  Dessa forma, quando o banco recebe várias requisições simultâneas em um mesmo dado os usuários são obrigados à esperarem em uma fila~\cite{cassandraguide}.

A necessidade de transformar os dados em tabelas causa um aumento na complexidade da operação pois requer o uso de complexos algoritmos de mapeamento e estrutura. Mesmo quando uma base de dados pode ser coberta pelo modelo relacional, as vezes as diversas garantias providas por esse modelo gera uma sobrecarga que não seria necessária para tarefas simples. O schema rigoroso pode ser pesado para aplicações que precisam de velocidade, como aplicações web e blogs que possuem diversos tipos de atributos. Textos, comentários, imagens, vídeos fonte, código e outras informações precisam ser armazenadas em diversas tabelas, e como as aplicações na web são muito ágeis, precisam ser amparadas por uma base de dados igualmente ágil e com um schema de fácil adaptação ~\cite{nosqlevaluation}.

O considerável aumento na quantidade de dados deve ser considerado por grandes empresas como Facebook, Amazon e Google. Além de tratar terabytes/petabytes de dados, realizar requisições de leitura e escrita na base a todo o momento essas empresas devem se preocupar com o tempo que essas transações estão levando, ou seja, a latência. Para tratar esses requisitos é preciso mantes milhares de máquinas com um hardware moderno e veloz. Por ter que cumprir com os requisitos de ACID e manter os dados normalizados, um modelo relacional não é adequado para esse cenário, visto que as operações de join bloqueiam os dados e influenciam negativamente no desempenho da aplicação.

Outro requisito fundamental para as grandes empresas é a disponibilidade de seus serviços. Para isso a base de dados deve ser facilmente replicável e fornecer uma forma automática de tratamento à falha de bases ou do datacenter. Esses SGBDs também devem ser capazes de balancear a carga em várias máquinas para não sobrecarregar um único servidor. Bancos relacionais priorizam a consistência em detrimento à disponibilidade e também possuem um mecanismo de replicação limitado.

Esses problemas podem ser resolvidos de algumas formas.Primeiramente optamos por um upgrade simples de hardware. Se o problema persistir a próxima opção seria adicionarmos novos servidores ao cluster, porém com os problemas de consistência e replicação durante o uso regular e em cenários de falha. A próxima etapa seria melhorar a configuração do gerenciador de banco de dados. Caso as opções de melhoria no SGBD se esgotem é preciso melhorar a aplicação. Verifica-se o desempenho das consultas, criamos índices e etc. Se o desempenho ainda não for satisfatório então talvez coloquemos uma camada de cache, mas que também gera um problema de consistência. Se mesmo assim o desempenho não atender as expectativas, então é necessário pensarmos novamente no SGBD. A última opção seria uma desnormarlização do banco, mas assim se estaria indo contra os princípios da modelagem relacional e das regras normais~\cite{cassandraguide}.

Dado toda essa problemática surge uma opção. Bancos de dados que não seguem o paradigma relacional. Os dados não são normalizados.

%\subsection{Bases Não Relacionais}
%\subsubsection{Teorema BASE}




