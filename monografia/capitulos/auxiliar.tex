\chapter{AUXILIAR}


\section{MongoDB}


Essa seção foi baseada no site oficial do MongoDB~\cite{sitemongodb} exceto quando explicitamente citado.

MongoDB é um banco de dados NoSQL,  de código aberto,  orientado a documentos, schema-free e escrito em C++.  Os dados são persistidos em coleções de dados que são representados usando o BSON, um formato binário similar ao JSON (Figura \ref{fig:exbson}). O MongoDB tem suporte a todos os tipos de dados  JSON  como string, inteiro, boleano, double, array e objeto. Por usar codificação BSON o MongoDB suporta alguns tipo de dados adicionais como data, binary data, regular expression e code ~\cite{nosqlprofessional}. Na Figura \ref{tab:bytes} podemos ver os tipos suportados pelo BSON.

\begin{table}
	\caption{BSON - Tipos Suportados}
	\begin{center}
	\begin{tabular}{ccc}
		\hline
			\textbf{Tipo} & \textbf{Número} \\
		\hline
			Double & 1 \\
			String & 2 \\
			Object & 3 \\
			Array & 4 \\
			Binary Data & 5 \\
			Object id & 7 \\
			Boolean & 8 \\
			Date & 9 \\
			Null & 10 \\
			Regular Expression & 11 \\
			JavaScript & 13 \\
			Symbol & 14 \\
			JavaScript (with scope) & 15 \\
			32-bit integer & 16 \\
			Timestamp& 17 \\
			64-bit integer & 18 \\
			Min key & 255 \\
			Max key & 127 \\
		\hline
	\end {tabular}
	\end{center}
	%\caption{Fonte: http://docs.mongodb.org}
	\label{tab:bsontypes}
\end{table}

Como não usa o mesmo formato de armazenamento dos SGBDS relacionais, o MongoDB armazena os seus dados em coleções, que são equivalentes às tabelas.  Uma Coleção pode ter um ou mais documentos; são equivalentes as linhas em uma tabela de um banco de dados relacional. Cada documento tem um ou mais campos, o que corresponde a uma coluna.

Diferente do que a maioria das pessoas estão acostumadas, o MongoDB não trabalha com uma estrutura de dados bem definida (schema), ou melhor dizendo, ele usa schemas dinâmicos. Com ele é possível criar coleções sem que a estrutura, campos ou tipos de valores dos documentos estejam definidos. Essa forma flexível de armazenar os dados nos permite trabalhar com estruturas e dados bastante heterogêneos.

	\begin{figure}[!htbp]
		\begin{center}
			\includegraphics[width=1.2\textwidth]{exbson}
		\end{center}
		\caption{Documento BSON usado no MongoDB ~\cite{sitemongodb}}
		\label{fig:exbson}
	\end{figure}

Quanto mais controle, mais custosa é uma operação para o sistema gerenciador de banco de dados.  Os banco de dados NoSQL, como dito anteriormente, foram criados para suprir algumas características que os banco de dados relacionais não atendiam. Uma dessas características é a velocidade com que operações de consulta, escrita, atualização e exclusão são executadas.  Para que a velocidade dessas transações fosse aumentada, foi preciso retirar alguns controles e, com isso, os banco de dados NoSQL não se comprometem com todas as características ACID.

O MongoDB não provê transações ACID, mas possui alguns recursos transacionais básicos. Operações atômicas são possíveis no escopo de um único documento. Na tabela abaixo temos alguns exemplos de operações em SQL e suas correspondentes no MongoDB.

\subsection{Modelagem dos Dados}

Cada documento tem um campo chamado ID que é utilizado como chave primária. Para aumentar a velocidade das queries é possível habilitar índices para os campos que são utilizados nas consultas. O MongoDB também suporta índices em sub-documentos e em arrays.

Ao contrário dos bancos de dados convencionais o MongoDB possui um schema flexível e não nos força a determinar uma estrutura antes de inserir os dados. As coleções no MongoDB não nos impedem de evoluir a estrutura dos documentos ~\cite{Orendanalysisand}.

Para representar as relações entre os objetos temos duas estratégias: referências e sub-documentos ~\cite{Orendanalysisand}.

\subsubsection{Referências}

Essa estratégia representa as relações entre os dados incluindo links ou referências de um documento para outro. Para acessar os dados referenciados a aplicação deve resolver a referência. Na figura \ref{fig:referencia}, temos um exemplo de modelagem utilizando referência.

	\begin{figure}[!htbp]
		\begin{center}
			\includegraphics[width=0.8\textwidth]{referencia}
		\end{center}
		\caption{ Modelagem utilizando referência. Adaptado de ~\cite{sitemongodb}}
		\label{fig:referencia}
	\end{figure}

Seguem algumas ocasiões em que podemos utilizar essa estratégia de modelagem:

\begin{itemize}
	\item Quando, ao criar sub-documentos, criamos duplicação de dados e essa duplicação não nos dá um ganho de performance vantajoso.
	\item Quando desejamos representar complexas relações de n para n.
	\item Para representar grandes um grande volume de dados de forma hierárquica.
\end{itemize}

Ao implementar as referências temos mais flexibilidade se compararmos com os sub-documentos. Em contrapartida, ao realizarmos consultas, essas referências deverão ser traduzidas, o que gera mais consultas ao servidor ~\cite{Orendanalysisand}.

\subsubsection{Sub-Documentos}

A outra estratégia possível é o uso de sub-documentos. Ao utilizar essa forma de armazenamento os dados relacionados são armazenados em um único documento. O MongoDB permite armazenar essas relações em sub-documentos ou arrays de documentos. Veja um exemplo desse tipo de modelagem na figura \ref{fig:subdocumento}.

	\begin{figure}[!htbp]
		\begin{center}
			\includegraphics[width=0.8\textwidth]{subdocumento}
		\end{center}
		\caption{ Modelagem utilizando sub-documento. Adaptado de ~\cite{sitemongodb}}
		\label{fig:subdocumento}
	\end{figure}

Seguem algumas ocasiões em que podemos utilizar os sub-documentos:

\begin{itemize}
	\item Em relações um para um;
	\item Em relações um para muitos. Nesses relacionamentos o objeto que se repete deve estar contido no outro objeto;
\end{itemize}

Os sub-documentos aumentam a performance de operações de leitura e nos dá a vantagem de obter os dados necessários em uma simples consulta ao banco. Com os sub-documentos também é possível realizar atualizações de forma atômica ~\cite{Orendanalysisand}.

\subsection{Linguagem de Consulta}

A sintaxe da linguagem de consulta do MongoDB é similar ao JSON. A linguagem de consulta permite consultar todos os documentos em uma coleção, inclusive os sub-documentos e os arrays ~\cite{Orendanalysisand}.

A linguagem de consulta suporta ~\cite{Orendanalysisand}]:
\begin{itemize}
	\item Consultas em documentos e sub-documentos
	\item Comparações
	\item Operações lógicas
	\item Ordenação por múltiplos campos
	\item Group by
	\item Uma agregação por consulta
\end{itemize}

Adicionalmente o MongoDB permite realizar consultas com agregações mais complexas utilizando uma variação do MapReduce ~\cite{Orendanalysisand}.

Nas tabelas \ref{tab:sqlvsmongo} e  \ref{tab:sqlvsmongoselect} temos algumas comparações entre a linguagem utilizada pelo MongoDB e o SQL.

\begin{table}[h]
	\caption{Declarações SQL vs Declarações MongoDB. Adaptado de ~\cite{sitemongodb}}
	\begin{center}
	\begin{tabular}{  l   p{8cm} }
		\hline
			\textbf{SQL} & \textbf{MongoDB} \\
		\hline
\lstset{language=SQL}
\begin{lstlisting}
CREATE TABLE users (
	id MEDIUMINT NOT NULL
		AUTO_INCREMENT,
	user_id Varchar(30),
	age Number,
	status char(1),
	PRIMARY KEY (id)
)
\end{lstlisting}
 & O documento é criado na primeira operação de inserção. Se o campo id não for especificado ele é automaticamente gerado.
\lstset{language=Java}
\begin{lstlisting}
db.users.insert( {
    user_id: "abc123",
    age: 55,
    status: "A"
 } )
\end{lstlisting}
\\ \hline
\lstset{language=SQL}
\begin{lstlisting}
ALTER TABLE users
ADD join_date DATETIME
\end{lstlisting}
 & O MongoDB não amarra a estrutura das coleções. Não existe alteração estrutural no nivel das coleções. As alterações ocorrem no nível dos documentos.
\lstset{language=Java}
\begin{lstlisting}
db.users.update(
    { },
    { $set: { join_date: new Date() } },
    { multi: true }
)
\end{lstlisting}
\\ \hline
\lstset{language=SQL}
\begin{lstlisting}
ALTER TABLE users
DROP COLUMN join_date
\end{lstlisting}
&
\lstset{language=Java}
\begin{lstlisting}
db.users.update(
    { },
    { $unset: { join_date: "" } },
    { multi: true }
)
\end{lstlisting}
\\ \hline
\lstset{language=SQL}
\begin{lstlisting}
DROP TABLE users
\end{lstlisting}
&
\lstset{language=Java}
\begin{lstlisting}
db.users.drop()
\end{lstlisting}
\\ \hline
	\end {tabular}
	\end{center}
	\label{tab:sqlvsmongo}
\end{table}

\begin{table}[h]
	\caption{SQL Select vs MongoDB Select. Adaptado de ~\cite{sitemongodb}}
	\begin{center}
	\begin{tabular}{  l   p{8cm} }
		\hline
			\textbf{SQL} & \textbf{MongoDB} \\
		\hline
\lstset{language=SQL}
\begin{lstlisting}
SELECT *
FROM users
\end{lstlisting}
 &
\lstset{language=Java}
\begin{lstlisting}
db.users.find()
\end{lstlisting}
\\ \hline
\lstset{language=SQL}
\begin{lstlisting}
SELECT id, user_id, status
FROM users
\end{lstlisting}
 &
\lstset{language=Java}
\begin{lstlisting}
db.users.find(
    { },
    { user_id: 1, status: 1 }
)
\end{lstlisting}
\\ \hline
\lstset{language=SQL}
\begin{lstlisting}
SELECT *
FROM users
WHERE status = "A"
\end{lstlisting}
&
\lstset{language=Java}
\begin{lstlisting}
db.users.find(
    { status: "A" }
)
\end{lstlisting}
\\ \hline
\lstset{language=SQL}
\begin{lstlisting}
SELECT *
FROM users
WHERE status = "A"
AND age = 50
\end{lstlisting}
&
\lstset{language=Java}
\begin{lstlisting}
db.users.find(
    { status: "A",
      age: 50 }
)
\end{lstlisting}
\\ \hline
\lstset{language=SQL}
\begin{lstlisting}
SELECT COUNT(*)
FROM users
\end{lstlisting}
&
\lstset{language=Java}
\begin{lstlisting}
db.users.count()
\end{lstlisting}

ou

\begin{lstlisting}
db.users.find().count()
\end{lstlisting}
\\ \hline
\lstset{language=SQL}
\begin{lstlisting}
EXPLAIN SELECT *
FROM users
WHERE status = "A"
\end{lstlisting}
&
\lstset{language=Java}
\begin{lstlisting}
db.users.find( { status: "A" } ).explain()
\end{lstlisting}
\\ \hline
	\end {tabular}
	\end{center}
	\label{tab:sqlvsmongoselect}
\end{table}



\subsection{Replicação}

Replicação é o processo de sincronizar dados através de diferentes servidores. Com a replicação é possível prover redundância e aumentar a disponibilidade dos dados, além de proteger os dados de uma possível falha de hardware ou catástrofes.

Um conjunto de réplicas é um grupo de instâncias do MongoDB com os mesmos dados. Na arquitetura de um conjunto de replicação somente um servidor, o primário, recebe todas as requisições de escrita. Os outros servidores somente replicam as operações em suas instâncias. A figura \ref{fig:replication} representa a arquitetura para replicação.

Como somente um nó recebe todas as operações de escrita, para suportar a replicação, o nó primário armazena em log todas as operações. Os nós secundarios replicam os logs do nó primário e, em seguida, realizam as operações em suas instâncias. Caso o nó primário fique indisponível, o conjunto de replicação eleje um novo nó para ser o primário. Por padrão, as requisições de leitura são feitas ao nó primário, porém isso pode ser alterado. Como a replicação é assíncrona, se as preferências de leitura forem alteradas os dados retornados podem não ser os mais atuais.

	\begin{figure}[!htbp]
		\begin{center}
			\includegraphics[width=0.5\textwidth]{replication}
		\end{center}
		\caption{Arquitetura de Replicação ~\cite{sitemongodb}}
		\label{fig:replication}
	\end{figure}


\subsection{Decomposição (Sharding)}

Sharding, ou decomposição, é quando separamos a localização fisica dos dados, despedaçando cada informação e colocando-as em nós diferentes. Essa técnica é utilizada para trabalhar com dados de grande volume e com alta vazão de operações. A decomposição é a alternativa à scalabilidade vertical. No MongoDB a decomposição é implementada com o uso de um cluster de decomposição. Os componentes desse cluster são: Shards, rotiadores de consulta e servidores de configuração (Figura \ref{fig:sharding}).

Os shards armazenam os dados.

Os rotiadores de consulta, ou mongos, se comunicam com os clientes e direcionam as operações ao shard ou shards apropriados.

Os servidores de conifguração armazenam os metadados do cluster. Ele contém o mapa do cluster. O roteador de consulta utiliza esse componente para encontrar os shards que serão utilizados.

	\begin{figure}[!htbp]
		\begin{center}
			\includegraphics[width=0.5\textwidth]{sharding}
		\end{center}
		\caption{Cluster de Sharding ~\cite{sitemongodb}}
		\label{fig:sharding}
	\end{figure}

\subsection{Tratamento de Falhas}

O MongoDB não utiliza log de transações para garantir a durabilidade dos dados. E por utilizar arquivos mapeados em memória, implementa escrita preguiçosa (lazy write). Sendo asssim, se um nó MongoDB falhar, provavelmente algum dado será perdido ~\cite{Orendanalysisand}.














