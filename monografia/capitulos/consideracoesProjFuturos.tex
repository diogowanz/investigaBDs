\chapter{Considerações Finais e Projeto Futuros}

Nesse capítulo faremos algumas considerações finais sobre o projeto e também serão apresentados alguns possíveis trabalhos futuros.

\section{Considerações Finais}

É claro que o MongoDB já se figura entre os mais populares bancos de dados NoSQL e que o PostgreSQL já é uma solução consolidada. O nosso trabalho visou testar a performance dessas duas soluções em um cenário definido, mas ambos possuem aplicação nas mais diversas soluções feitas no mundo. Como exemplo temos a implantação do MongoDB no CartolaFC ~\cite{mongocartola} da Globo.com e, em relação ao PostgreSQL, no site oficial ~\cite{usecasepostgresql}  temos diversos casos de sucesso.

\begin{itemize}
	\item O código fonte da aplicação, \textit{scripts} utilizados e todos os planos de teste podem ser encontrados no repositório do projeto ~\cite{github}.
\end{itemize}

\section{Projetos Futuros}

Existem ainda outros diversos testes que podem enriquecer os estudos sobre as tecnologias apresentadas nesse trabalho. A seguir está enumerado três possíveis projetos:

\begin{enumerate}
\item Modelar a estrutura do MongoDB usando somente sub-documentos e verificar qual o impacto na performance;
\item Implementar a arquitetura em uma infraestrutura mais robusta, com mais de uma máquina executando o JMeter e realizando requisições distribuídas, além de realizar os testes com uma massa de dados maior. Na tabela \ref{tab:infra1}, \ref{tab:infra2} e \ref{tab:infra3} temos uma sugestão de infra-estrutura para a realização de testes;
\item Expandir esses testes para outros bancos de dados como o MySQL e Cassandra.
\end{enumerate}

\begin{table}[h]
	\caption{Infra-Estrutura para Trabalhos Futuros}
	\begin{center}
	\begin{tabular} {|l|l|}
		\hline
			\multicolumn{2}{|c|}{SERVIDOR 1 - APLICAÇÃO - QTD: 1} \\
		\hline
			Sistema Operacional & Debian 6.0 - Squeeze com interface gráfica \\
		\hline
			\multicolumn{2}{|c|}{SOFTWARES}\\
		\hline
			Python & Versão 2.6.6 com pymongo e psycopg2\\
		\hline
			PostgreSQL & 8.4.16\\
		\hline
			MongoDB & 2.4\\
		\hline
			web2py & 2.5.1\\
		\hline
			Servidor Web Apache & 2.2\\
		\hline
			\multicolumn{2}{|c|}{HARDWARE}\\
		\hline
			Disco & 30 TB\\
		\hline
			Memória & 8 GB\\
		\hline
	\end {tabular}
	\end{center}
	\label{tab:infra1}
\end{table}

\begin{table}[h]
	\caption{Infra-Estrutura para Trabalhos Futuros}
	\begin{center}
	\begin{tabular} {|l|l|}
		\hline
			\multicolumn{2}{|c|}{JMETER MÁSTER - QTD: 1} \\
		\hline
			Sistema Operacional & Debian 6.0 - Squeeze com interface gráfica \\
		\hline
			\multicolumn{2}{|c|}{SOFTWARES}\\
		\hline
			JMeter & 2.9\\
		\hline
			\multicolumn{2}{|c|}{HARDWARE}\\
		\hline
			Disco & 200 GB\\
		\hline
			Memória & 6 GB\\
		\hline
	\end {tabular}
	\end{center}
	\label{tab:infra2}
\end{table}


\begin{table}[h]
	\caption{Infra-Estrutura para Trabalhos Futuros}
	\begin{center}
	\begin{tabular} {|l|l|}
		\hline
			\multicolumn{2}{|c|}{JMETER - CLIENTES - QTD: 5} \\
		\hline
			Sistema Operacional & Debian 6.0 - Squeeze com interface gráfica \\
		\hline
			\multicolumn{2}{|c|}{SOFTWARES}\\
		\hline
			JMeter & 2.9\\
		\hline
			\multicolumn{2}{|c|}{HARDWARE}\\
		\hline
			Disco & 200 GB\\
		\hline
			Memória & 4 GB\\
		\hline
	\end {tabular}
	\end{center}
	\label{tab:infra3}
\end{table}


